% Options for packages loaded elsewhere
\PassOptionsToPackage{unicode}{hyperref}
\PassOptionsToPackage{hyphens}{url}
%
\documentclass[
]{article}
\usepackage{amsmath,amssymb}
\usepackage{lmodern}
\usepackage{iftex}
\ifPDFTeX
  \usepackage[T1]{fontenc}
  \usepackage[utf8]{inputenc}
  \usepackage{textcomp} % provide euro and other symbols
\else % if luatex or xetex
  \usepackage{unicode-math}
  \defaultfontfeatures{Scale=MatchLowercase}
  \defaultfontfeatures[\rmfamily]{Ligatures=TeX,Scale=1}
\fi
% Use upquote if available, for straight quotes in verbatim environments
\IfFileExists{upquote.sty}{\usepackage{upquote}}{}
\IfFileExists{microtype.sty}{% use microtype if available
  \usepackage[]{microtype}
  \UseMicrotypeSet[protrusion]{basicmath} % disable protrusion for tt fonts
}{}
\makeatletter
\@ifundefined{KOMAClassName}{% if non-KOMA class
  \IfFileExists{parskip.sty}{%
    \usepackage{parskip}
  }{% else
    \setlength{\parindent}{0pt}
    \setlength{\parskip}{6pt plus 2pt minus 1pt}}
}{% if KOMA class
  \KOMAoptions{parskip=half}}
\makeatother
\usepackage{xcolor}
\usepackage[margin=1in]{geometry}
\usepackage{color}
\usepackage{fancyvrb}
\newcommand{\VerbBar}{|}
\newcommand{\VERB}{\Verb[commandchars=\\\{\}]}
\DefineVerbatimEnvironment{Highlighting}{Verbatim}{commandchars=\\\{\}}
% Add ',fontsize=\small' for more characters per line
\usepackage{framed}
\definecolor{shadecolor}{RGB}{248,248,248}
\newenvironment{Shaded}{\begin{snugshade}}{\end{snugshade}}
\newcommand{\AlertTok}[1]{\textcolor[rgb]{0.94,0.16,0.16}{#1}}
\newcommand{\AnnotationTok}[1]{\textcolor[rgb]{0.56,0.35,0.01}{\textbf{\textit{#1}}}}
\newcommand{\AttributeTok}[1]{\textcolor[rgb]{0.77,0.63,0.00}{#1}}
\newcommand{\BaseNTok}[1]{\textcolor[rgb]{0.00,0.00,0.81}{#1}}
\newcommand{\BuiltInTok}[1]{#1}
\newcommand{\CharTok}[1]{\textcolor[rgb]{0.31,0.60,0.02}{#1}}
\newcommand{\CommentTok}[1]{\textcolor[rgb]{0.56,0.35,0.01}{\textit{#1}}}
\newcommand{\CommentVarTok}[1]{\textcolor[rgb]{0.56,0.35,0.01}{\textbf{\textit{#1}}}}
\newcommand{\ConstantTok}[1]{\textcolor[rgb]{0.00,0.00,0.00}{#1}}
\newcommand{\ControlFlowTok}[1]{\textcolor[rgb]{0.13,0.29,0.53}{\textbf{#1}}}
\newcommand{\DataTypeTok}[1]{\textcolor[rgb]{0.13,0.29,0.53}{#1}}
\newcommand{\DecValTok}[1]{\textcolor[rgb]{0.00,0.00,0.81}{#1}}
\newcommand{\DocumentationTok}[1]{\textcolor[rgb]{0.56,0.35,0.01}{\textbf{\textit{#1}}}}
\newcommand{\ErrorTok}[1]{\textcolor[rgb]{0.64,0.00,0.00}{\textbf{#1}}}
\newcommand{\ExtensionTok}[1]{#1}
\newcommand{\FloatTok}[1]{\textcolor[rgb]{0.00,0.00,0.81}{#1}}
\newcommand{\FunctionTok}[1]{\textcolor[rgb]{0.00,0.00,0.00}{#1}}
\newcommand{\ImportTok}[1]{#1}
\newcommand{\InformationTok}[1]{\textcolor[rgb]{0.56,0.35,0.01}{\textbf{\textit{#1}}}}
\newcommand{\KeywordTok}[1]{\textcolor[rgb]{0.13,0.29,0.53}{\textbf{#1}}}
\newcommand{\NormalTok}[1]{#1}
\newcommand{\OperatorTok}[1]{\textcolor[rgb]{0.81,0.36,0.00}{\textbf{#1}}}
\newcommand{\OtherTok}[1]{\textcolor[rgb]{0.56,0.35,0.01}{#1}}
\newcommand{\PreprocessorTok}[1]{\textcolor[rgb]{0.56,0.35,0.01}{\textit{#1}}}
\newcommand{\RegionMarkerTok}[1]{#1}
\newcommand{\SpecialCharTok}[1]{\textcolor[rgb]{0.00,0.00,0.00}{#1}}
\newcommand{\SpecialStringTok}[1]{\textcolor[rgb]{0.31,0.60,0.02}{#1}}
\newcommand{\StringTok}[1]{\textcolor[rgb]{0.31,0.60,0.02}{#1}}
\newcommand{\VariableTok}[1]{\textcolor[rgb]{0.00,0.00,0.00}{#1}}
\newcommand{\VerbatimStringTok}[1]{\textcolor[rgb]{0.31,0.60,0.02}{#1}}
\newcommand{\WarningTok}[1]{\textcolor[rgb]{0.56,0.35,0.01}{\textbf{\textit{#1}}}}
\usepackage{graphicx}
\makeatletter
\def\maxwidth{\ifdim\Gin@nat@width>\linewidth\linewidth\else\Gin@nat@width\fi}
\def\maxheight{\ifdim\Gin@nat@height>\textheight\textheight\else\Gin@nat@height\fi}
\makeatother
% Scale images if necessary, so that they will not overflow the page
% margins by default, and it is still possible to overwrite the defaults
% using explicit options in \includegraphics[width, height, ...]{}
\setkeys{Gin}{width=\maxwidth,height=\maxheight,keepaspectratio}
% Set default figure placement to htbp
\makeatletter
\def\fps@figure{htbp}
\makeatother
\setlength{\emergencystretch}{3em} % prevent overfull lines
\providecommand{\tightlist}{%
  \setlength{\itemsep}{0pt}\setlength{\parskip}{0pt}}
\setcounter{secnumdepth}{-\maxdimen} % remove section numbering
\ifLuaTeX
  \usepackage{selnolig}  % disable illegal ligatures
\fi
\IfFileExists{bookmark.sty}{\usepackage{bookmark}}{\usepackage{hyperref}}
\IfFileExists{xurl.sty}{\usepackage{xurl}}{} % add URL line breaks if available
\urlstyle{same} % disable monospaced font for URLs
\hypersetup{
  pdftitle={Lab 4 - Introduction to RStudio and R Markdown. Presentations},
  pdfauthor={Ranjitprakash Sundaramurthi},
  hidelinks,
  pdfcreator={LaTeX via pandoc}}

\title{Lab 4 - Introduction to RStudio and R Markdown. Presentations}
\author{Ranjitprakash Sundaramurthi}
\date{Oct 1, 2022}

\begin{document}
\maketitle

{
\setcounter{tocdepth}{2}
\tableofcontents
}
\begin{center}\rule{0.5\linewidth}{0.5pt}\end{center}

In Lab 4, you will learn how to use, edit and create a R Markdown
document (like this one) using RStudio. You should follow the
instructions in this document to complete the assignment. Knit this
document to view the nicely rendered HTML, which can make it easier to
read the questions.

If you need help as you use R Markdown in this lab and others in the
future, consult the following resources:

\begin{itemize}
\tightlist
\item
  \href{https://rmarkdown.rstudio.com/lesson-15.html}{Cheat sheet}
\item
  \href{https://rmarkdown.rstudio.com/docs/}{Home page with guides}
\item
  \href{https://bookdown.org/yihui/rmarkdown/}{Reference book}
\end{itemize}

The below is a code chunk, but instead of using the \texttt{r} engine
we're creating and alert block that will make the question show up with
a blue background in the HTML output. Unfortunately, this creates and
error when exporting to PDF, so it can only be used for HTML.

\begin{alert alert-info}
\hypertarget{submission-instructions}{%
\subsection{Submission Instructions}\label{submission-instructions}}

rubric=\{mechanics:2\}

You receive mark for submitting your lab correctly, please follow these
instructions:

\begin{itemize}
\tightlist
\item
  \href{https://ubc-mds.github.io/resources_pages/general_lab_instructions/}{Follow
  the general lab instructions}.
\item
  \href{https://github.com/UBC-MDS/public/tree/master/rubric}{Click here
  to view a description of the rubrics used to grade the questions}.
\item
  Push your \texttt{.Rmd} AND \textbf{all the files} you will create as
  part of the lab to your GitHub repository.

  \begin{itemize}
  \tightlist
  \item
    The reason for pushing all the files is that \texttt{.Rmd} does not
    contain the rendered output from running the cells. If someone is
    checking out your work there needs to be an HTML file to view the
    output, so it is good to get into this habit. - \texttt{.ipynb}
    renders nicely on GitHub, which is why we did not include the HTML
    file for previous labs.
  \end{itemize}
\item
  Upload a \texttt{.Rmd} version of your assignment to Gradescope.
\item
  Include a clickable link to your GitHub repo for the lab just below
  this cell (it should look something like this
  \url{https://github.ubc.ca/MDS-2022-23/DSCI_521_labX_yourcwl}.
\end{itemize}
\end{alert alert-info}

\url{https://github.ubc.ca/MDS-2022-23/DSCI_521_lab4_ranjit86}

\hypertarget{editing-r-markdown-documents}{%
\subsection{Editing R Markdown
documents}\label{editing-r-markdown-documents}}

This document is called an R Markdown document. It is a literate code
document, similar to Jupyter notebooks where you can write code and view
its outputs. To start, let's set our working directory by creating a new
R Project for lab 4.

\hypertarget{text-and-rendering-r-markdown-documents}{%
\subsubsection{Text and rendering R Markdown
documents}\label{text-and-rendering-r-markdown-documents}}

In a R Markdown document any line of text not in a code chunk (like this
line of text) will be formatted using Markdown. Similar to JupyterLab,
you can also use HTML and LaTeX here to do more advanced formatting. To
run a code chunk, you can press the green play button in the top right
corner of the chunk.

\begin{alert alert-info}
\hypertarget{question-1}{%
\paragraph{Question 1}\label{question-1}}

rubric=\{correctness:1\}

To render the HTML files we are going to create, the first step is to
activate GitHub pages.
\end{alert alert-info}

\href{https://ranjitprakash1986.github.io/new-repo-lab4/}{\textbf{https://ranjitprakash1986.github.io/new-repo-lab4/}}

\begin{alert alert-info}
\hypertarget{question-2}{%
\paragraph{Question 2}\label{question-2}}

rubric=\{mechanics:1\}

Create a new code chunk below using the r language engine that runs some
R code (it does not need to be complicated, but it should have an
output). Ensure that you can render/knit the document after you add that
chunk.
\end{alert alert-info}

\hypertarget{code-chunk-to-add-two-numbers}{%
\paragraph{Code chunk to add two
numbers}\label{code-chunk-to-add-two-numbers}}

\begin{Shaded}
\begin{Highlighting}[]
\NormalTok{a }\OtherTok{\textless{}{-}} \DecValTok{10}
\NormalTok{b }\OtherTok{\textless{}{-}} \DecValTok{5}
\FunctionTok{print}\NormalTok{(a }\SpecialCharTok{+}\NormalTok{ b)}
\end{Highlighting}
\end{Shaded}

\begin{verbatim}
## [1] 15
\end{verbatim}

\begin{alert alert-info}
\hypertarget{question-3}{%
\paragraph{Question 3}\label{question-3}}

rubric=\{mechanics:1\}

Create a new code chunk, and add a meaningful name to the code chunk.
Try using the pop-up-like menu to navigate between the named code chunks
Don't forget to knit/render the document after you make this change to
ensure everything is still working.
\end{alert alert-info}

\hypertarget{code-chunk-to-multiply-two-numbers}{%
\paragraph{Code Chunk to multiply two
numbers}\label{code-chunk-to-multiply-two-numbers}}

\begin{Shaded}
\begin{Highlighting}[]
\NormalTok{x }\OtherTok{=} \DecValTok{5}
\NormalTok{y }\OtherTok{=} \DecValTok{3}
\FunctionTok{print}\NormalTok{ (x }\SpecialCharTok{*}\NormalTok{ y)}
\end{Highlighting}
\end{Shaded}

\begin{verbatim}
## [1] 15
\end{verbatim}

\begin{alert alert-info}
\hypertarget{question-4}{%
\paragraph{Question 4}\label{question-4}}

rubric=\{mechanics:1,reasoning:1\}

Create a new code chunk that uses a code chunk option. Write out in your
own words what the code chunk option is doing.
\end{alert alert-info}

\hypertarget{code-chunk-showing-radius-vs-area-of-circle-relationship-and-chunk-options}{%
\paragraph{Code Chunk showing Radius vs Area of Circle relationship and
chunk
options}\label{code-chunk-showing-radius-vs-area-of-circle-relationship-and-chunk-options}}

\begin{Shaded}
\begin{Highlighting}[]
\NormalTok{r }\OtherTok{\textless{}{-}} \FunctionTok{c}\NormalTok{(}\FloatTok{0.5}\NormalTok{, }\DecValTok{1}\NormalTok{, }\FloatTok{1.5}\NormalTok{, }\FloatTok{2.0}\NormalTok{, }\FloatTok{2.5}\NormalTok{)}
\NormalTok{area }\OtherTok{\textless{}{-}} \DecValTok{2}\SpecialCharTok{*}\NormalTok{pi}\SpecialCharTok{*}\NormalTok{r}\SpecialCharTok{\^{}}\DecValTok{2}
\FunctionTok{plot}\NormalTok{(r, area,}\AttributeTok{xlab =} \StringTok{\textquotesingle{}Radius\textquotesingle{}}\NormalTok{, }\AttributeTok{ylab =} \StringTok{\textquotesingle{}Area\textquotesingle{}}\NormalTok{,}\AttributeTok{type =} \StringTok{\textquotesingle{}l\textquotesingle{}}\NormalTok{, }\AttributeTok{main =} \StringTok{\textquotesingle{}Radius vs Area plot\textquotesingle{}}\NormalTok{)}
\end{Highlighting}
\end{Shaded}

\includegraphics{lab4_files/figure-latex/Answer4-1.pdf}

echo: (True/False) This option allows us to print the source code to be
evaluated. A value of `True' shows the R code and `False' hides the R
code. Here TRUE is chosen and hence the source code as well as the plot
are seen in the knitted output .

eval: (True/False), False is used when we want to show the code but do
not want it to run while knitting the markdown. Here, I want to compute
the plot and hence eval is selected as TRUE.

\hypertarget{multiple-code-chunk-options}{%
\paragraph{Multiple code chunk
options}\label{multiple-code-chunk-options}}

To have multiple code chunk options you separate them by a comma. For
example, if in addition to suppressing warnings, we want to run the code
but not output the results, then we can add the
\texttt{include\ =\ FALSE} argument to the code chunk after the
\texttt{warning\ =\ FALSE} option.

\begin{alert alert-info}
\hypertarget{question-5}{%
\paragraph{Question 5}\label{question-5}}

rubric=\{mechanics:1,reasoning:1\}

Create a new code chunk that uses at least two code chunk options. At
least one must be different to the ones mentioned above. Write in your
own words what each code chunk option is doing.
\end{alert alert-info}

\begin{Shaded}
\begin{Highlighting}[]
\NormalTok{a }\OtherTok{\textless{}{-}} \DecValTok{20}
\NormalTok{b }\OtherTok{\textless{}{-}} \DecValTok{100}
\NormalTok{a}\SpecialCharTok{*}\NormalTok{b}
\end{Highlighting}
\end{Shaded}

\begin{verbatim}
## [1] 2000
\end{verbatim}

The include and highlight code chunk options are used here:

Include: (TRUE/FALSE) This option allows us to include or exclude the
chunk output in the outputted document. Here as it is indicated as TRUE,
the result of the multiplication (2000) is shown as output.

results: (markup/asis) This option allows us to control how to display
the results.

\hypertarget{yaml-header-and-document-output-options}{%
\subsubsection{1.5. YAML Header and document output
options}\label{yaml-header-and-document-output-options}}

R Markdown files contains three types of content:

\begin{enumerate}
\def\labelenumi{\arabic{enumi}.}
\tightlist
\item
  Plain text mixed with simple Markdown formatting.
\item
  Code chunks surrounded by ```.
\item
  An (optional) YAML header surrounded by \texttt{-\/-\/-}.
\end{enumerate}

You have been introduced the first two types of content, but not the
third (although you probably saw it at the top of this document). The
(optional) YAML header, which is located at the very top of R Markdown
files sets some general global parameters, including:

\begin{itemize}
\tightlist
\item
  title
\item
  author
\item
  output
\item
  etc
\end{itemize}

\textbf{Example YAML Header}

\begin{verbatim}
---
title: "Reproducible Data Science Report"
author: "Florencia D'Andrea"
date: "September 4, 2022"
output: html_document
---
\end{verbatim}

Most important from a workflow perspective is \textbf{output}. Possible
output options include:

\begin{itemize}
\tightlist
\item
  \texttt{output:\ html\_document}
\item
  \texttt{output:\ md\_document}
\item
  \texttt{output:\ pdf\_document}
\item
  \texttt{output:\ word\_document}
\item
  \texttt{output:\ xaringan::moon\_reader} (xaringan presentation -
  html)
\end{itemize}

\begin{alert alert-info}
\hypertarget{question-6}{%
\paragraph{Question 6}\label{question-6}}

rubric=\{mechanics:1\}

Navigate to the YAML header at the very top of this document and edit it
so that you include an \texttt{author} (yourself) and a \texttt{date}
(lab due date). Include what you added below here as well as a fenced
Markdown code block.
\end{alert alert-info}

\begin{alert alert-info}

\end{alert alert-info}

\hypertarget{creating-r-markdown-documents}{%
\subsubsection{Creating R Markdown
documents}\label{creating-r-markdown-documents}}

You can use the ``File'' menu inside RStudio to create new R Markdown
documents by selecting: \textbf{File \textgreater{} New File
\textgreater{} R Markdown} This will bring you to another menu where you
can choose the type of output (don't be afraid to pick something, you
can always change the \texttt{output} type once you have the
\texttt{.Rmd} file).

To create a written report, we generally recommend using the default
\texttt{output:\ html\_document} as it is easier to read than PDF (note
- LaTeX does not render nicely in such documents sadly, so if you are
using a lot of LaTeX then you may want to choose
\texttt{output:\ pdf\_document}). If you want to create an \texttt{.md}
file to publish on GitHub, it is recommend to instead use
\texttt{output:\ github\_document}. To get this from the menu above you
need to navigate to the ``From Template'' option on the left panel and
then select ``GitHub Document (Markdown)''.

\begin{alert alert-info}
\hypertarget{question-7}{%
\paragraph{Question 7}\label{question-7}}

rubric=\{mechanics:2\}

1 - Create a new RMarkdown report (a different file than this one) in
the same directory as this RMarkdown file. Use \texttt{html\_document}
as the \texttt{output}. After you have rendered it, paste the link to
the HTML output as a link to your GitHub repository (remember to push
all your files!)

2 - Then, navigate to the YAML header at the very top of that
\texttt{.Rmd} document and edit it so that the \texttt{output} is
\texttt{pdf\_document}. Then knit/render the document. Note the
different output. Add and commit that rendered both the \texttt{.html}
and \texttt{.pdf} files to the GitHub repository for this lab and paste
the two links below this question.
\end{alert alert-info}

\begin{enumerate}
\def\labelenumi{\arabic{enumi}.}
\tightlist
\item
  \url{https://github.ubc.ca/MDS-2022-23/DSCI_521_lab4_ranjit86/blob/master/Sample_HTML_Document.html}
\end{enumerate}

(OR)
\url{https://github.com/ranjitprakash1986/new-repo-lab4/blob/master/Sample_HTML_Document.html}

\begin{enumerate}
\def\labelenumi{\arabic{enumi}.}
\setcounter{enumi}{1}
\item
  \textbf{PDF Links:}
  \url{https://github.ubc.ca/MDS-2022-23/DSCI_521_lab4_ranjit86/blob/master/Sample_HTML_Document.pdf}

  (OR)
  \url{https://github.com/ranjitprakash1986/new-repo-lab4/blob/master/Sample_HTML_Document.pdf}

  \textbf{HTML Links:}
  \url{https://github.ubc.ca/MDS-2022-23/DSCI_521_lab4_ranjit86/blob/master/Sample_HTML_Document.html}

  (OR)
  \url{https://github.com/ranjitprakash1986/new-repo-lab4/blob/master/Sample_HTML_Document.html}

  \textbf{RMD Links:}
  \url{https://github.ubc.ca/MDS-2022-23/DSCI_521_lab4_ranjit86/blob/master/Sample_HTML_Document.Rmd}

  (OR)
  \url{https://github.com/ranjitprakash1986/new-repo-lab4/blob/master/Sample_HTML_Document.Rmd}
\end{enumerate}

\begin{alert alert-info}
\hypertarget{question-8}{%
\paragraph{Question 8}\label{question-8}}

rubric=\{mechanics:6\}

\begin{enumerate}
\def\labelenumi{\arabic{enumi}.}
\item
  Go back to the \texttt{.Rmd} file you created in question 7, and
  include at least two Markdown text sections (each should have a
  header) and at least two separate code chunks in it (these can be
  really simple). Save the new R Markdown document and give it a new
  meaningful name.
\item
  Render/knit the new R Markdown document to get an \texttt{.html} file.
  Put the \texttt{.Rmd} document and the rendered \texttt{.html} file
  under version control using Git, and push/upload the file to your
  GitHub repository for this homework. Paste a link to these files as
  your answer below.
\end{enumerate}
\end{alert alert-info}

\begin{enumerate}
\def\labelenumi{\arabic{enumi}.}
\tightlist
\item
  See the document links below for the answer.
\end{enumerate}

\begin{enumerate}
\def\labelenumi{\arabic{enumi}.}
\setcounter{enumi}{1}
\item
  \textbf{Rmd Links:}
  \url{https://github.ubc.ca/MDS-2022-23/DSCI_521_lab4_ranjit86/blob/master/Computing_area_perimeter.Rmd}

  (OR)
  \url{https://github.com/ranjitprakash1986/new-repo-lab4/blob/master/Computing_area_perimeter.Rmd}

  \textbf{Html Links:}
  \url{https://github.ubc.ca/MDS-2022-23/DSCI_521_lab4_ranjit86/blob/master/Computing_area_perimeter.html}

  (OR)
  \url{https://github.com/ranjitprakash1986/new-repo-lab4/blob/master/Computing_area_perimeter.html}
\end{enumerate}

\begin{alert alert-info}
\hypertarget{question-9-optional}{%
\paragraph{Question 9 (Optional)}\label{question-9-optional}}

rubric=\{mechanics:1,reasoning:1\}

\begin{enumerate}
\def\labelenumi{\arabic{enumi}.}
\tightlist
\item
  Take the R Markdown report created in Question 8 and change the output
  to \texttt{github\_document} and render it. Put the rendered
  \texttt{.md} file under version control using Git, and push/upload the
  file to your GitHub repository for this homework. Try to look at the
  file on GitHub.ubc.ca in your homework repo? What do you see? How is
  it rendered?
\end{enumerate}
\end{alert alert-info}

\textbf{md Links:}
\url{https://github.ubc.ca/MDS-2022-23/DSCI_521_lab4_ranjit86/blob/master/Computing_area_perimeter.md}

(OR)
\url{https://github.com/ranjitprakash1986/new-repo-lab4/blob/master/Computing_area_perimeter.md}

The file is rendered for viewing in the github repository directly. It
closely follows the appearance of a html page with slight differences in
the font. On the other hand the HTML version of the document is not
rendered directly on the repo, it shows the inherent html code when
opened.

\begin{alert alert-info}
\hypertarget{question-10}{%
\paragraph{Question 10}\label{question-10}}

rubric=\{mechanics:6\}

\begin{enumerate}
\def\labelenumi{\arabic{enumi}.}
\tightlist
\item
  Create a presentation using RStudio. Do this in a different file than
  this one but in the same directory as this RMarkdown file. You can use
  xaringan or Quarto to create the slides.
  \href{https://pages.github.ubc.ca/MDS-2022-23/DSCI_521_platforms-dsci_students/materials/lectures/8-rmarkdown-quarto-slides-ghpages.html}{On
  the book} you will find links that will guide you on how to create
  each type of slide. If you are not sure which one to use, it is safer
  to use Xaringan as Quarto is quite new and you will have to learn how
  to use \texttt{.qmd} files. But if you want to explore Quarto, we will
  accept both options (Xaringan or Quarto slides) as correct. Give this
  file a meaningful name.
\item
  Create at least 4 slides. At least two slides must include a code
  chunk or cell (these can be really simple). Save the new document.
\item
  Render/knit the new document to get a \texttt{html} presentation file.
\item
  Put the new document and the rendered \texttt{.html} file under
  version control using Git, and push/upload the file to your GitHub
  repository for this lab
\item
  Activate GitHub pages and paste the link below.
\end{enumerate}
\end{alert alert-info}

\textbf{Github page:}

\url{https://ranjitprakash1986.github.io/new-repo-lab4/Math_operators.html\#1}

\textbf{RMDLinks:}
\url{https://github.ubc.ca/MDS-2022-23/DSCI_521_lab4_ranjit86/blob/master/Math_operators.Rmd}

(OR)
\url{https://github.com/ranjitprakash1986/new-repo-lab4/blob/master/Math_operators.Rmd}

\textbf{HTML Links:}
\url{https://github.ubc.ca/MDS-2022-23/DSCI_521_lab4_ranjit86/blob/master/Math_operators.html}

(OR)
\url{https://github.com/ranjitprakash1986/new-repo-lab4/blob/master/Math_operators.html}

\begin{alert alert-info}
\hypertarget{challenging-question-11}{%
\paragraph{(Challenging) Question 11}\label{challenging-question-11}}

rubric=\{reasoning\}

In a paragraph or two, compare and contrast the use of reproducible
tools (e.g., R Markdown and Jupyter) and non-reproducible tools (Word,
Powerpoint, Keynote, etc) for presentations and reports. Include
advantages and disadvantages for each.
\end{alert alert-info}

YOUR ANSWER GOES HERE

\end{document}
